% Copyright 2020 by Junwei Wang <i.junwei.wang@gmail.com>
%
% This file may be distributed and/or modified under the
% conditions of the LaTeX Project Public License, either version 1.3c
% of this license or (at your option) any later version.
% The latest version of this license is in
%   http://www.latex-project.org/lppl.txt

%\documentclass[compress]{beamer}
\documentclass[landscape,a4paper,]{beamer}

\newcommand\myheading[1]{%
  \par\bigskip
  {\Large\bfseries#1}\par\smallskip}

\newcommand\mysubheading[1]{%
  \par\bigskip
  {\Medium\bfseries#1}\par\smallskip}


\useoutertheme{infolines}
\graphicspath{{Images/}}

\usepackage[english]{babel}
\usepackage{metalogo}
\usepackage{listings}
\usepackage{fontspec}
\usepackage{tikz}

\usepackage{wrapfig}
\usetheme{Nord}
% \usetheme[style=light]{Nord}

\setmainfont{Yanone Kaffeesatz}
\setsansfont{Andika New Basic}
\setmonofont{DejaVu Sans Mono}

\usepackage[utf8]{inputenc}
\usepackage[T1]{fontenc}
\usepackage{graphicx}
\usepackage{grffile}
\usepackage{longtable}
\usepackage{wrapfig}
\usepackage{rotating}
\usepackage[normalem]{ulem}
\usepackage{amsmath}
\usepackage{textcomp}
\usepackage{amssymb}
\usepackage{capt-of}
\usepackage{hyperref}
\author{Saksham Sharma}
\date{\today}
\title{Computer Practical Project 2021}
\hypersetup{
 pdfauthor={Saksham Sharma},
 pdftitle={Computer Practical Project 2021},
 pdfkeywords={},
 pdfsubject={},
 pdfcreator={Emacs 27.2 (Org mode 9.5)}, 
 pdflang={English}}
\begin{document}

\maketitle
\tableofcontents
\newpage

\section{Overview}
\label{sec:orga8bb4a8}
\myheading{\insertsectionhead}

\begin{wrapfigure}{r}{0.25\textwidth}
    \includegraphics[scale=0.08]{Images/library.png}
    \label{fig:wrapfig}
\end{wrapfigure}

Following is a document listing the features of a Library Management System that was built after being frustrated by the manual handling of data at the local library. The features include a system to manage a database of books, Admins and Clients, and the ability to modify, sort and export relevant data.
The document also includes basic instructions if one wants to modify the program

\section{Objective}
\label{sec:orgbc0d5a6}
\myheading{\insertsectionhead}
\begin{wrapfigure}{r}{0.40\textwidth}
    \includegraphics[scale=0.12]{Login_Page.png}
    \label{fig:wrapfig}
\end{wrapfigure}
To Build a Library management system that is easy to use and provides a hassle free experience
\begin{itemize}
\item Built with \href{https://www.python.org/}{Python} and \href{https://www.qt.io/}{QT Framework}
\item MySQL Database Hosted on \href{//azure.microsoft.com}{Azure}
\end{itemize}

\newpage

\section{Features}
\label{sec:org94e4192}
\myheading{\insertsectionhead}

\begin{wrapfigure}{r}{0.50\textwidth}
    \includegraphics[scale=0.1]{Main.png}
    \label{fig:wrapfig}
\end{wrapfigure}

\subsection{Main system}
\label{sec:orgbd99333}
\mysubheading{\insertsubsectionhead}
\begin{itemize}
\item Login Page
\item Users
\item Add Users (Signup)
\item Add a Book
\item Edit Book Info
\item Delete Book
\item Categories
\item Settings [Categs, Author, Publisher]
\item Day To Day Transaction Log
\item Reports [Excel Files]
\end{itemize}

\newpage

\begin{wrapfigure}{r}{0.50\textwidth}
    \includegraphics[scale=0.1]{Books.png}
    \label{fig:wrapfig}
\end{wrapfigure}
\subsection{Book}
\label{sec:org56fa84e}
\mysubheading{\insertsubsectionhead}


\begin{itemize}
\item Title
\item ISBN Code
\item Description
\item Category
\item Price
\item Author
\item Publisher
\end{itemize}

\begin{wrapfigure}{r}{0.50\textwidth}
    \includegraphics[scale=0.11]{Users.png}
    \label{fig:wrapfig}
\end{wrapfigure}

\subsection{Admin}
\label{sec:org8df7786}
\mysubheading{\insertsubsectionhead}


\begin{itemize}
\item User Name
\item Password
\item Email Id
\end{itemize}

\begin{wrapfigure}{r}{0.50\textwidth}
    \includegraphics[scale=0.11]{Client.png}
    \label{fig:wrapfig}
\end{wrapfigure}

\subsection{Client / Students}
\label{sec:org8ca5e45}
\mysubheading{\insertsubsectionhead}


\begin{itemize}
\item Username
\item Student ID
\item Email ID
\end{itemize}

\begin{wrapfigure}{r}{0.50\textwidth}
    \includegraphics[scale=0.11]{Theme.png}
    \label{fig:wrapfig}
\end{wrapfigure}

\subsection{Day-to-Day}
\label{sec:org1c6a657}
\mysubheading{\insertsubsectionhead}

\begin{itemize}
\item Book name
\item Type (Issue / Return)
\item Duration (weeks)
\end{itemize}

\subsection{Category, Publisher, Author}
\label{sec:org2ea43dc}
\mysubheading{\insertsubsectionhead}


\begin{itemize}
\item Names
\item List
\end{itemize}

\section{Future Prospects}
\label{sec:org4d396ab}
\myheading{\insertsectionhead}
\begin{itemize}
\item WebApp
\item Integrate barcode scanning
\item Ability to select local or cloud database
\end{itemize}

\newpage
\section{Resources}
\label{sec:org7bd709f}
\myheading{\insertsectionhead}

\subsection{Requirements}
\label{sec:org0ceccbb}
\mysubheading{\insertsubsectionhead}

\subsubsection{QtDesigner}
\label{sec:org7bc29de}
If you want to edit the .ui files ie the actual LOOK or LAYOUT
\begin{itemize}
\item \href{https://build-system.fman.io/static/public/files/Qt\%20Designer\%20Setup.exe}{\textbf{\textbf{Install on Windows (Link)}}}
\item \href{https://build-system.fman.io/static/public/files/Qt\%20Designer.dmg}{\textbf{\textbf{Install on Mac (Link)}}}
\begin{block}{Installing on Linux (Debian and Fedora)}
%\item \textbf{\textbf{Installing on Linux (Debian and Fedora)}}
sudo apt-get install qttools5-dev-tools qttools5-dev

sudo dnf install qttools5-dev-tools qttools5-dev

\end{block}
\end{itemize}
\subsubsection{Python Dependencies}
\label{sec:org03faeef}
To install python dependencies use the \texttt{requirements.txt} file with python pip

\subsection{Icon Manipulation:}
\label{sec:orgd388344}
\mysubheading{\insertsubsectionhead}

Follow these steps to change hte programs icons
\begin{enumerate}
\item Keep the same name (for ease)
\begin{exampleblock}{In the folder, open terminal and run the following code}
   \text{pyrcc5 icons.qrc -o icons\_rc.py}
\end{exampleblock}
\end{enumerate}
\end{document}
