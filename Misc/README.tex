% Created 2021-09-23 Thu 10:49
% Intended LaTeX compiler: pdflatex
\documentclass[11pt]{article}
\usepackage[utf8]{inputenc}
\usepackage[T1]{fontenc}
\usepackage{graphicx}
\usepackage{grffile}
\usepackage{longtable}
\usepackage{wrapfig}
\usepackage{rotating}
\usepackage[normalem]{ulem}
\usepackage{amsmath}
\usepackage{textcomp}
\usepackage{amssymb}
\usepackage{capt-of}
\usepackage{hyperref}
\author{Saksham Sharma}
\date{\textit{<2021-09-22 Wed>}}
\title{Computer Practical Project 2021}
\hypersetup{
 pdfauthor={Saksham Sharma},
 pdftitle={Computer Practical Project 2021},
 pdfkeywords={},
 pdfsubject={Project documentation for Computer},
 pdfcreator={Emacs 27.2 (Org mode 9.5)}, 
 pdflang={English}}
\begin{document}

\maketitle
\tableofcontents

\textbf{\textbf{Name:}} Library Management system
\textbf{\textbf{Version:}} 1.0

\section{Overview}
\label{sec:org83c6530}
Following is a document listing the features of a Library Management System that was built after being frustrated by the manual handling of data at the local library. The features include a system to manage a database of books, Admins and Clients, and the ability to modify, sort and export relevant data.
The document also includes basic instructions if one wants to modify the program

\section{Objective}
\label{sec:orgd7d8056}
\begin{itemize}
\item Built with \href{https://www.python.org/}{Python} and \href{https://www.qt.io/}{QT Framework}
\item MySQL Database Hosted on \href{//azure.microsoft.com}{Azure}
\end{itemize}

\section{Features}
\label{sec:org6a543fb}
\subsection{Main system}
\label{sec:orgf987840}
\begin{itemize}
\item Login Page
\item Users
\item Add Users (Signup)
\item Add a Book
\item Edit Book Info
\item Delete Book
\item Categories
\item Settings [Categs, Author, Publisher]
\item Day To Day Transaction Log
\item Reports [Excel Files]
\end{itemize}

\subsection{Book}
\label{sec:orge1f6131}
\begin{itemize}
\item Title
\item ISBN Code
\item Description
\item Category
\item Price
\item Author
\item Publisher
\end{itemize}

\subsection{Admin}
\label{sec:orgae946ac}
\begin{itemize}
\item User Name
\item Password
\item Email Id
\end{itemize}

\subsection{Client / Students}
\label{sec:org1ea0c7f}
\begin{itemize}
\item Username
\item Student ID
\item Email ID
\end{itemize}

\subsection{Day-to-Day}
\label{sec:org8918086}
\begin{itemize}
\item Book name
\item Type (Issue / Return)
\item Duration (weeks)
\end{itemize}

\subsection{Category, Publisher, Author}
\label{sec:orga00290e}
\begin{itemize}
\item Names
\item List
\end{itemize}

\section{Future Prospects}
\label{sec:org5002023}
\begin{itemize}
\item WebApp
\item Integrate barcode scanning
\item Ability to select local or cloud database
\end{itemize}

\section{Resources}
\label{sec:orgf0c1647}
\subsection{Requirements}
\label{sec:orgbeb5504}

\subsubsection{QtDesigner}
\label{sec:orgec9a439}
If you want to edit the .ui files ie the actual LOOK or LAYOUT
\begin{itemize}
\item \href{https://build-system.fman.io/static/public/files/Qt\%20Designer\%20Setup.exe}{\textbf{\textbf{Install on Windows}}}
\item \href{https://build-system.fman.io/static/public/files/Qt\%20Designer.dmg}{\textbf{\textbf{Install on Mac}}}
\item \textbf{\textbf{Installing on Linux (Debian and Fedora)}}
\end{itemize}
\begin{verbatim}
sudo apt-get install qttools5-dev-tools qttools5-dev

sudo dnf install qttools5-dev-tools qttools5-dev
\end{verbatim}

\subsubsection{Python Dependencies}
\label{sec:org00694f9}
To install python dependencies use the \texttt{requirements.txt} file with python pip

\subsection{Icon Manipulation:}
\label{sec:orgce07cb3}
Follow these steps to change hte programs icons
\begin{enumerate}
\item Keep the same name (for ease)
\item In the folder, open terminal and run the following code
\begin{verbatim}
   pyrcc5 icons.qrc -o icons_rc.py
\end{verbatim}
\end{enumerate}
\end{document}
